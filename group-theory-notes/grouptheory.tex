\documentclass[11pt]{article}
\usepackage[utf8]{inputenc}
\usepackage{amsfonts}
\usepackage{amsmath}
\usepackage{amssymb}
\usepackage{amsthm}
\usepackage{geometry}
\setlength{\parindent}{0pt}
\setlength{\parskip}{8pt}
\usepackage{graphicx}
\usepackage{multicol}
\usepackage{mathrsfs}
\usepackage[shortlabels]{enumitem}
\usepackage{hyperref}
\usepackage{siunitx}
\usepackage{stmaryrd}
\usepackage{physics}


\newcommand{\R}{\mathbb{R}}
\newcommand{\QED}{\rightline{\emph{Quod erat demonstrandum.}}}
\newcommand{\xhat}{\mathbf{\hat{x}}}
\newcommand{\yhat}{\mathbf{\hat{y}}}
\newcommand{\zhat}{\mathbf{\hat{z}}}
\newcommand{\thetahat}{\mathbf{\hat{\theta}}}
\newcommand{\phihat}{\mathbf{\hat{\phi}}}
\newcommand{\rhat}{\mathbf{\hat{r}}}
\newcommand{\shat}{\mathbf{\hat{s}}}
\def\rcurs{{\mbox{$\resizebox{.16in}{.08in}{\includegraphics{images/ScriptR.pdf}}$}}}
\def\brcurs{{\mbox{$\resizebox{.16in}{.08in}{\includegraphics{images/BoldR.pdf}}$}}}
\def\hrcurs{{\mbox{$\hat \brcurs$}}}

\allowdisplaybreaks


\title{Group Theory Notes}
\author{Harsh Jaluka}
\date{}

\begin{document}

\begin{titlepage}
\maketitle 
\end{titlepage}

\newpage 
\tableofcontents

\newpage
\section{List of groups with definitions and representations}

\subsection{$SO(3): \{R \in Aut(\mathbb{V}), R^TR = 1, det(R) = 1 \}$}
\begin{itemize}
    \item ``Special Orthogonal Group'' (``3D rotation group")
    \item $SO(3) \subset Aut(\R^3)$
    \item Fundamental/defining/vector representation is the identity map
    \item Lie algebra $ \mathfrak{so}(3) = \{ r \in End(\mathbb{V}); r^T = -r \}$
        \begin{itemize}
            \item an (imaginary) basis: $J_k := i\vec{e}^\times_k \in i \mathfrak{so}(3) \subset \mathfrak{so}(3, \mathbb{C})$ 
            \item defining representation $\rho_{\mathfrak{so}(3)}(J_k) = i \vec{e}_k^\times \in i \mathfrak{so}(3) \subset iEnd(\mathbb{R}^3)$ 
            \item Lie bracket expansion: $\llbracket J_i, J_j \rrbracket = i \epsilon_{ijk}J_k$ 
            \item $N$-dimensional irrep labelled by non-negative `spin' $j = \frac{1}{2}(N-1)$
            \begin{itemize}
                \item representation space spanned by 
                \begin{align*}
                    \ket{m},~~~~~ m \in \{-j, -j +1, \cdots, j -1, j \}
                \end{align*}
                \item action of generators $J_z, J_{\pm} = J_x \pm i J_y$ 
                \begin{align*}
                    \rho(J_z) \ket{m} &= m \ket{m} \\
                    \rho(J_\pm) \ket{m} &= c_m^\pm \ket{m \pm 1}
                \end{align*}
                where 
                \begin{align*}
                    c_m^\pm = \sqrt{(j \mp m)(j \pm m +1)}
                \end{align*}
                \item unitarity 
                \begin{align*}
                    \rho(J_z)^\dagger = \rho(J_z), ~~~~ \rho(J_+)^\dagger = \rho(J_-)
                \end{align*}
                \item application: spherical harmonics (TODO)
                \item character polynomial
                \begin{align*}
                    P_j(q) = \sum_{k=0}^{2j} q^{2k-2j} = \frac{q^{2j+1}- q^{-2q-1}}{q-q^{-1}}
                \end{align*}
                \item the tensor product of two representations with spin $j$ and $j'$ yields 
                \begin{align*}
                    P_{j \otimes j'} (q) = P_j(q) P_{j'}(q) &= \frac{q^{2j+1} -q^{-2j-1} }{q - q^{-1}}\frac{q^{2j'+1} - q^{-2j' - 1}}{q - q^{-1}} \\
                    &= \sum_{k=0}^{2j'}P_{j+j'-k}(q) 
                \end{align*}w
                in the last step, assume $j' \leq j$. If not, $P_{-k} = -P_{k-1}$
            \end{itemize}
            \item Casimir invariant $C = J_k \otimes J_k = J_z \otimes J_z + \frac{1}{2}J_+ \otimes J_- + \frac{1}{2}J_- \otimes J_+$ with 
            \begin{align*}
                \rho(C) \ket{m} = j(j+1) \ket{m}
            \end{align*}
        \end{itemize}
\end{itemize}

\subsection{$SU(2)$}
\begin{itemize}
    \item double cover of $SO(3)$
    \item Lie algebra $\mathfrak{su}(2) = \{m \in End(\mathbb{C}^2); m = -m^\dagger, tr(m) = 0\}$ 
        \begin{itemize}
            \item an (imaginary) basis: $J_k = \frac{1}{2}\sigma_k \in i \mathfrak{su}(2)$ where $\sigma_k$ Pauli matrices
            \item defining representation $\rho_{\mathfrak{su}(2)} (J_k) = \frac{1}{2}\sigma_k \in i \mathfrak{su}(2) \subset End(\mathbb{C}^2)$
            \item Lie bracket expansion: $\llbracket J_i, J_j \rrbracket = i\epsilon_{ijk} J_k$ 
        \end{itemize}
\end{itemize}

\subsection{$Sp(1)$}
\begin{itemize}
    \item Lie algebra $\mathfrak{sp}(1)$ 
        \begin{itemize}
            \item defining representation $\rho_{\mathfrak{sp}(1)}(-iJ_x) = \frac{1}{2}\vu{i}$ and similarly $J_y \to \vu{j}$ and $J_z \to \vu{k}$
            \begin{itemize}
                \item symplectic
                \begin{align*}
                    \rho_{\mathfrak{sp}(1)}(a)^\dagger = - \rho_{\mathfrak{sp}(1)}(a)
                \end{align*}
            \end{itemize}
        \end{itemize}
\end{itemize}

\subsection{$S_3$}
\begin{itemize}
    \item $|S_3| = 3! = 6$ elements 
    \item all elements can be written in terms of two elementary permutations $\sigma_1, \sigma_2$ which satisfy $(\sigma_1)^2 = (\sigma_2)^2 = (\sigma_1\sigma_2)^2 = 1$
    \item can be viewed as symmetry group of an equilateral triangle
    \item  
\end{itemize}

\subsection{Other groups}

\begin{enumerate}
    \item $SO(2)$: orthogonal matrices with unit determinant
    \item (Reals, +) 
    \item $O(2)$ 
    \begin{itemize}
        \item dihedral group is a subgroup 
    \end{itemize}
	\item (Reals/{0}, x) 
	\item $S_n$, $n>1$
        \begin{itemize}
            \item ``symmetric group''
            \item all $n!$ permutations of a set of $n$ elements
        \end{itemize}
	\item U(1): set of unitary 1x1 matrices (unitary: conjugate transpose = inverse)
    \item $G^(1) = {R_\phi^(1): \phi \in R/2piZ}$  
	\item $Aut(\R^3)$
	\item Spin(3)
    \item $O(3) = \{ R \in Aut(\R^3); R^TR = 1\}$
        \begin{itemize}
            \item ``group of reflections''
            \item Lie algebra $\mathfrak{so}(3) =\{ r \in End(\mathbb{V}); r^T = -r \}$ 
        \end{itemize}
    \item $C_n = \mathbb{Z}_N = \mathbb{Z}/n\mathbb{Z}$, $n>1$ 
        \begin{itemize}
            \item ``cyclic group''
            \item integers modulo $n$ under addition
            \item abelian
        \end{itemize}
    \item $SU(2)_L \times SU(2)_R$
        \begin{itemize}
            \item double cover of $SO(4)$
        \end{itemize}
    \item $\{ e\}$ 
        \begin{itemize}
            \item ``trivial group''
        \end{itemize}
    \item $A_n$, $n>1$
        \begin{itemize}
            \item ``alternating group''
            \item all $n!/2$ even permutations of a set of $n$ elements
        \end{itemize}
\end{enumerate}

\section{Definitions}
\begin{itemize}
    \item Lie group: a group whose set $G$ is a differentiable manifold and whose composition rule and inversion are smooth maps on this manifold 
    \item universal cover of a group: a bigger group which contains the original group and is simply connected
    \item abelian: ab = ba
    \item automorphism: invertible linear transformation from $\mathbb{V}$ to itself 
    \item endomorphism: linear map from $\mathbb{V}$ to itself 
    \item composition rule: group operation (eg addition for group of reals equipped with +)
    \item For any Lie group there is a corresponding Lie algebra with a Lie bracket
        \begin{itemize}
            \item tangent space of a group at the identity 
            \item not associative 
            \item when the Lie algebra is given in terms of matrices, i.e. $\mathfrak{g} \subset End(\mathbb{V})$, the matrix commutator and the Lie bracket coincide. 
        \end{itemize}
    \item quaternions $\mathbb{H}$: number field spanned by $1, \vu{i}, \vu{j}, \vu{k}$ with 
        \begin{itemize}
            \item products given by $\vu{i} \vu{j} = - \vu{j} \vu{i} = \vu{k}$ and cyclic permutations ($\implies$ non commutativity)
            \item closely related to Pauli matrices 
            \begin{align*}
                (1, \vu{i}, \vu{j}, \vu{k}) \equiv (1, -i\sigma_x, -i\sigma_y, -i\sigma_z)
            \end{align*}
        \end{itemize}
    \item Casimir operator $C \in \mathfrak{g} \otimes \mathfrak{g}$ with $\llbracket a, C \rrbracket = 0, \forall a \in \mathfrak{g}$ where 
    \begin{align*}
        \llbracket a, b \otimes c \rrbracket := \llbracket a,b \rrbracket \otimes c + b \otimes \llbracket a,c \rrbracket
    \end{align*}
    \begin{itemize}
        \item any representation $\rho: \mathfrak{g} \to End(\mathbb{V})$ can be lifted to $\rho: \mathfrak{g} \otimes \mathfrak{g} \to End(\mathbb{V})$ with 
        \begin{align*}
            \rho (a \otimes b) := \rho(a) \rho(b)
        \end{align*}
    \end{itemize}
    \item tensor product of Lie algebra representations $\rho_\otimes: \mathfrak{g} \to End(\mathbb{V_\otimes})$ defined as 
    \begin{align*}
        \rho_\otimes := \sum_{k=1}^N 1 \otimes \cdots \otimes 1 \otimes \rho_k \otimes 1 \otimes \cdots \otimes 1 
    \end{align*}
    where $\rho_k: \mathfrak{g} \to End(\mathbb{V}_k), k = 1, \cdots, N$ 
    \item direct sum of Lie algebra representations $\rho_\oplus: \mathfrak{g} \to End(\mathbb{V}_\oplus)$ defined as 
    \begin{align*}
        \rho_\oplus := \rho_1 \oplus \cdots \oplus \rho_N 
    \end{align*}
    \item character polynomials: define a group element $g(q)$ depending on a formal variable $q$ 
    \begin{align*}
        g(q) = \exp(2\log(q)J_z) = q^{2J_z}
    \end{align*}
    define the character as 
    \begin{align*}
        P_\rho (q) = \tr \rho(g(q)) = \sum_k n_k q^{2m_k}
    \end{align*}
    where $m_k$ eigenvalues of $J_z$ and $n_k$ corresponding multiplicities 
    \begin{itemize}
        \item $q = 1$ gives dimension
        \item have the property
         \begin{align*}
            P_{\rho \otimes \rho'}(q) &= P_\rho(q) P_{\rho'}(q) \\
            P_{\rho \oplus \rho'}(q) &= P_\rho(q) + P_{\rho'}(q)
        \end{align*}
        \item 
    \end{itemize}

\end{itemize}

\section{Isomorphisms}
\begin{itemize}
    \item $\mathfrak{so}(3) \equiv \mathfrak{su}(2) \equiv \mathfrak{sp}(1)$
    \item $Spin(3) \equiv SU(2) \equiv Sp(1) \equiv S^3 \equiv D_3$
    \item $SO(3) \equiv SU(2)/\mathbb{Z}_2$
    \item icosahedral $\equiv A_5$ 
    \item $T \equiv A_4$ 
    \item $T_d \equiv O \equiv S_4$ 
\end{itemize}

\section{Notation}
\begin{itemize}
    \item $\mathfrak{gothic}$ script represents Lie algebras
    \item $\vec{v}^\times$ denotes the $3\times3$ anti-symmetric matrix that defines the cross product of $\vec{v}$ with an arbitrary vector $\vec{w}$ via matrix multiplication: $\vec{v}^\times \vec{w} = \vec{v} \times \vec{w}$
        \begin{align*}
            \vec{v}^\times := 
            \begin{bmatrix}
                0 & -v_z & + v_y \\
                +v_z & 0 & -v_x \\
                -v_y & + v_x & 0 
            \end{bmatrix}
        \end{align*}
\end{itemize}

\section{Maps}
\begin{itemize}
    \item adjoint action: $Ad(R): \mathfrak{g} \to \mathfrak{g}$ such that $Ad(R)(a_1) := Ra_1 R^{-1} =: a_2$ with $a_1, a_2 \in \mathfrak{g}$ 
    \item adjoint representation: $Ad: G \to Aut(\mathfrak{g})$
    \item adjoint action $ad: \mathfrak{g} \to End(\mathfrak{g})$ is a linear map such that 
        \begin{itemize}
            \item $ad(a_2) a_1 = a_2a_1 - a_1a_2$
        \end{itemize}
    \item  ``Lie bracket'' $\llbracket , \rrbracket: \mathfrak{g} \times \mathfrak{g} \to \mathfrak{g}$
        \begin{itemize}
            \item anti-symmetric: $\llbracket a,b \rrbracket = - \llbracket b, a \rrbracket$ 
            \item Jacobi identity satisfied: $\llbracket a, \llbracket b,c \rrbracket \rrbracket + \llbracket b, \llbracket c,a \rrbracket \rrbracket + \llbracket c, \llbracket a,b \rrbracket \rrbracket = 0 $  
        \end{itemize}
        \item exponential map $\exp: \mathfrak{g} \to G$ 
            \begin{itemize}
                \item If matrix group $G \subset Aut(\mathbb{V})$ or if representation $\rho: G \to Aut(\mathbb{V})$, it coincides with the matrix exponential $\exp: End(\mathbb{V}) \to Aut(\mathbb{V})$
                    \begin{align*}
                        \exp a = \sum_{n=0}^\infty \frac{1}{n!} a^n
                    \end{align*}
                \item $\exp a \exp b = exp C(a,b)$ where $C(a,b)$ Baker-Campbell-Hausdorff formula: 
                    \begin{align*}
                        C(a,b) = a + b + \frac{1}{2}\llbracket a, b \rrbracket + \frac{1}{12} \llbracket a, \llbracket a,b \rrbracket \rrbracket + \frac{1}{12}\llbracket b, \llbracket b, a \rrbracket \rrbracket + \cdots 
                    \end{align*}
                \item $(\exp a)^{-1} = \exp (-a) $
            \end{itemize}
        \item Lie algebra representation $\rho: \mathfrak{g} \to End(\mathbb{V})$ such that 
        \begin{align*}
            \rho(\llbracket a,b \rrbracket) = [\rho(a), \rho(b)]
        \end{align*}
            \begin{itemize}
                \item trivial representation $\rho_0: \mathfrak{g} \to End(\mathbb{V})$ with $\rho_0 (J_k) = 0$
                \item adjoint representation $ad: \mathfrak{g} \to End(\mathfrak{g})$ with $\llbracket a,b \rrbracket = ad(a)b$ 
            \end{itemize}
\end{itemize}

\end{document}  